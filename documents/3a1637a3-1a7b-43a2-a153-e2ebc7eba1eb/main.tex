% !TEX program = xelatex
\documentclass{article}
\usepackage{fontspec}
\usepackage[russian]{babel}
\setmainfont{DejaVu Serif}
\setsansfont{DejaVu Sans}
\setmonofont{DejaVu Sans Mono}
\begin{document}
\title{производная}
\author{}
\date{}
\maketitle
\section{Введение}
Привет мир, производная функции - это математическое понятие, которое описывает скорость изменения функции в данной точке.

\section{Определение производной}
Определение производной функции $f(x)$ в точке $x=a$:
\[f'(a) = \lim_{h \to 0} \frac{f(a+h) - f(a)}{h}\]

\section{Основные теоремы}
Теорема 1: если функция $f(x)$ дифференцируема в точке $x=a$, то она непрерывна в этой точке.

\section{Примеры применения}
Пример 1: найти производную функции $f(x) = x^2$.
\[f'(x) = 2x\]
\end{document}