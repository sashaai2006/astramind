\documentclass{article}
\usepackage[utf8]{inputenc}
\usepackage[russian]{babel}
\title{производные}
\begin{document}
\maketitle
\section{Введение}
Текст параграфа.

Еще текст.
\subsection{Определение производной}
Определение производной функции.

Геометрический смысл производной функции.
\subsection{Геометрический смысл производной}
Геометрический смысл производной функции.

Производная функции представляет собой скорость изменения функции.
\section{Производные в машинном обучении}
Производные функций используются в машинном обучении.

Для оптимизации функций.
\subsection{Градиентный спуск}
Градиентный спуск - это метод оптимизации функций.

Используется для нахождения минимума функции.
\subsection{Оптимизация функций}
Оптимизация функций - это процесс нахождения минимума или максимума функции.

Используется в машинном обучении для обучения моделей.
\section{Применение производных в ML}
Производные функций используются в машинном обучении.

Для оптимизации функций и обучения моделей.
\subsection{Линейная регрессия}
Линейная регрессия - это метод обучения моделей.

Используется для прогнозирования значений.
\subsection{Нейронные сети}
Нейронные сети - это метод обучения моделей.

Используется для решения сложных задач.
\end{document}